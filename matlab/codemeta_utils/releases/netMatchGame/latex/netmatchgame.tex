\documentclass[12pt]{article}
\input{preamble/common}
\input{preamble/courseFormatting}
\usetikzlibrary{shapes.geometric}

%colors for the players
\definecolor{player1}{named}{RoyalBlue1}
\definecolor{player2}{named}{Firebrick1}
\definecolor{player3}{named}{Gold1}
\definecolor{player4}{named}{SpringGreen2}

%sizes of board elements
\newcommand{\placesize}{7mm}
\newcommand{\cardwidth}{3.33cm}
\newcommand{\cardheight}{\cardwidth*4/3}

%line and node styles for the board
\tikzset{link/.style={very thick},
  place/.style={circle,minimum size=\placesize,draw=black,line width=1mm},
  start/.style={fill=red,place},
  netnode/.style={fill=blue,place},
  polygon/.style={draw=orange,line width=2mm},
  card place/.style={polygon,align=center,minimum width=\cardwidth,minimum height=\cardheight},
}

%styles for pawn symbols
\tikzset{triang/.style={regular polygon, regular polygon sides=3},
  starplus/.style={star,star points=4,star point ratio=2.5}
}

\newcommand{\pawn}[3]{\node[circle,fill=player#1,minimum size=\placesize] at (#3) {};%
  \node[inner sep=0pt,fill=white,minimum size=0.8*\placesize,#2] at (#3) {};}



\newcommand{\gamename}{{\sffamily NetMatch: the Game}}

\title{\gamename{}  (version 0.1)}
\author{Roberto Tron}
\date{\today}
\begin{document}
\pagestyle{empty}

\maketitle

\section*{Overview}
The goal of \gamename{} is to discover your secret word by transferring and matching information across the network.
\section*{Game content}
\begin{itemize}
\item 64 feature pawns (4 copies of 4 symbols for each of the  4 colors).
\item 16 word tokens (4 symbols for each of the 4 colors).
\item 75 word cards.
\item 1 game board (with 1 dodecagonal network and 4 card boxes).
\end{itemize}
Note that all the pawns, tokens and cards need to be cut (using, e.g., scissors) from their respective sheets.

\section*{Rules}
\gamename{} is a game for 2-4 players. The rules below are for the 4 player version. See the corresponding section for the 3 players and 2 players variations.
\subsection*{Game preparation}
First, each player picks a color, and the deck of word cards is shuffled. Then, each player will setup the game for \emph{the next player on the right} using the following steps.
\begin{enumerate}
\item Pick a card and read the four-letter word on it, keeping it secret from other players.
\item Write each one of the letters under one of the \emph{other player's} letter token (not your own color).
\item Place the card (face down) in the \emph{other player's} card box on the board, and the four letter tokens on top of it.
\end{enumerate}
Finally, each player will place his own colored pawns on the sides of the dodecagonal network marked with their color; each side cannot contain the same symbol twice, and the symbols need to appear in the same order in each sector (see Figure~\ref{fig:pawn-arrangment}). Note that, in the 4 player game, only 48 of the pawns will be used.

\subsection*{Game play}
The youngest player goes first, and the turn is passed to the right.
\paragraph{Moves}
Each turn, the player can attempt the \emph{word gambit} (see below), or can move one of the pawns from a \emph{origin node} to a \emph{destination node} according to the following rules:
\begin{itemize}
\item The \emph{origin} and \emph{destination nodes} must be neighbors (i.e., directly connected with a black line).
\item The \emph{destination node} must be empty, or be occupied by a pawn with the same color and symbol.
\end{itemize}
If two pawns with the same color and symbol are in the same node, the player can declare a \emph{pairwise match}, and must remove one of the pawns involved from the board. If, by doing so, only one copy of a pawn remain in play (i.e., there have been two pairwise matches for that color/shape combination), the player can declare a \emph{multi-way match}, pick the corresponding letter token, and discover the letter written under it.
\paragraph{Winning}
A player can win by discovering all their four letters (by matching all the pawns and collecting all the tokens), which would allow them to read the word on their card.

Alternatively, a player can attempt, on his own turn, the \emph{word gambit} by stating a four letter word, and then turning their word card so that all players can see it: if the stated word and the word on the card match, then the player immediately wins; if they don't, the player immediately looses, and all their pawns are removed from the game.

\section*{Variations for two and three players}
There are two variations for playing the game with less than four players. In all cases, the game play is the same, only the game preparation is different.
\paragraph{Partial board variation} The same preparations as the four-players version are used, except that the colors that are not played are ignored. In this way, there are less pawns on the board, and there is less interaction between players.
\paragraph{Full board variation} For a three-player game, each player starts from four sides of the board instead of three. The sides should follow a cyclic alternating of colors (ignore the coloring already present on the board). For a two-player game, each player takes control of two colors; to win, a player must discover the words for both colors. With this variation, more matches are required to win.
\begin{figure}
\centering
\begin{tikzpicture}
  \begin{scope}[transform canvas={scale=0.2}]
  %This file has been generated by POCcostFunction.py

%Player sector colors
\path[fill=player1](8.8,0.0) -- (7.6210235533,4.4) -- (6.92820323028,4.0) -- (8.0,0.0) -- cycle;
\path[fill=player1](-4.4,7.6210235533) -- (-7.6210235533,4.4) -- (-6.92820323028,4.0) -- (-4.0,6.92820323028) -- cycle;
\path[fill=player1](-4.4,-7.6210235533) -- (-1.61653377487e-15,-8.8) -- (-1.46957615898e-15,-8.0) -- (-4.0,-6.92820323028) -- cycle;
\path[fill=player2](7.6210235533,4.4) -- (4.4,7.6210235533) -- (4.0,6.92820323028) -- (6.92820323028,4.0) -- cycle;
\path[fill=player2](-7.6210235533,4.4) -- (-8.8,1.07768918325e-15) -- (-8.0,9.79717439318e-16) -- (-6.92820323028,4.0) -- cycle;
\path[fill=player2](-1.61653377487e-15,-8.8) -- (4.4,-7.6210235533) -- (4.0,-6.92820323028) -- (-1.46957615898e-15,-8.0) -- cycle;
\path[fill=player3](4.4,7.6210235533) -- (5.38844591625e-16,8.8) -- (4.89858719659e-16,8.0) -- (4.0,6.92820323028) -- cycle;
\path[fill=player3](-8.8,1.07768918325e-15) -- (-7.6210235533,-4.4) -- (-6.92820323028,-4.0) -- (-8.0,9.79717439318e-16) -- cycle;
\path[fill=player3](4.4,-7.6210235533) -- (7.6210235533,-4.4) -- (6.92820323028,-4.0) -- (4.0,-6.92820323028) -- cycle;
\path[fill=player4](5.38844591625e-16,8.8) -- (-4.4,7.6210235533) -- (-4.0,6.92820323028) -- (4.89858719659e-16,8.0) -- cycle;
\path[fill=player4](-7.6210235533,-4.4) -- (-4.4,-7.6210235533) -- (-4.0,-6.92820323028) -- (-6.92820323028,-4.0) -- cycle;
\path[fill=player4](7.6210235533,-4.4) -- (8.8,0.0) -- (8.0,0.0) -- (6.92820323028,-4.0) -- cycle;
%Polygon
\draw[polygon] (8.0,0.0) -- (6.92820323028,4.0) -- (4.0,6.92820323028) -- (4.89858719659e-16,8.0) -- (-4.0,6.92820323028) -- (-6.92820323028,4.0) -- (-8.0,9.79717439318e-16) -- (-6.92820323028,-4.0) -- (-4.0,-6.92820323028) -- (-1.46957615898e-15,-8.0) -- (4.0,-6.92820323028) -- (6.92820323028,-4.0) -- cycle;
%Nodes and links
\path[start] (7.86602540378,0.5) coordinate[start] (s0) (7.59807621135,1.5) coordinate[start] (s1) (7.33012701892,2.5) coordinate[start] (s2) (7.06217782649,3.5) coordinate[start] (s3) (6.56217782649,4.36602540378) coordinate[start] (s4) (5.83012701892,5.09807621135) coordinate[start] (s5) (5.09807621135,5.83012701892) coordinate[start] (s6) (4.36602540378,6.56217782649) coordinate[start] (s7) (3.5,7.06217782649) coordinate[start] (s8) (2.5,7.33012701892) coordinate[start] (s9) (1.5,7.59807621135) coordinate[start] (s10) (0.5,7.86602540378) coordinate[start] (s11) (-0.5,7.86602540378) coordinate[start] (s12) (-1.5,7.59807621135) coordinate[start] (s13) (-2.5,7.33012701892) coordinate[start] (s14) (-3.5,7.06217782649) coordinate[start] (s15) (-4.36602540378,6.56217782649) coordinate[start] (s16) (-5.09807621135,5.83012701892) coordinate[start] (s17) (-5.83012701892,5.09807621135) coordinate[start] (s18) (-6.56217782649,4.36602540378) coordinate[start] (s19) (-7.06217782649,3.5) coordinate[start] (s20) (-7.33012701892,2.5) coordinate[start] (s21) (-7.59807621135,1.5) coordinate[start] (s22) (-7.86602540378,0.5) coordinate[start] (s23) (-7.86602540378,-0.5) coordinate[start] (s24) (-7.59807621135,-1.5) coordinate[start] (s25) (-7.33012701892,-2.5) coordinate[start] (s26) (-7.06217782649,-3.5) coordinate[start] (s27) (-6.56217782649,-4.36602540378) coordinate[start] (s28) (-5.83012701892,-5.09807621135) coordinate[start] (s29) (-5.09807621135,-5.83012701892) coordinate[start] (s30) (-4.36602540378,-6.56217782649) coordinate[start] (s31) (-3.5,-7.06217782649) coordinate[start] (s32) (-2.5,-7.33012701892) coordinate[start] (s33) (-1.5,-7.59807621135) coordinate[start] (s34) (-0.5,-7.86602540378) coordinate[start] (s35) (0.5,-7.86602540378) coordinate[start] (s36) (1.5,-7.59807621135) coordinate[start] (s37) (2.5,-7.33012701892) coordinate[start] (s38) (3.5,-7.06217782649) coordinate[start] (s39) (4.36602540378,-6.56217782649) coordinate[start] (s40) (5.09807621135,-5.83012701892) coordinate[start] (s41) (5.83012701892,-5.09807621135) coordinate[start] (s42) (6.56217782649,-4.36602540378) coordinate[start] (s43) (7.06217782649,-3.5) coordinate[start] (s44) (7.33012701892,-2.5) coordinate[start] (s45) (7.59807621135,-1.5) coordinate[start] (s46) (7.86602540378,-0.5) coordinate[start] (s47);
\path (-5.66458950116,-2.85371291234) coordinate[netnode] (n0) (-3.13953676463,-5.52085651601) coordinate[netnode] (n1) (2.28766112508,0.678831079224) coordinate[netnode] (n2) (5.70027506402,1.67571087853) coordinate[netnode] (n3) (0.162584255511,-2.39218233072) coordinate[netnode] (n4) (0.453712276129,-6.3681749908) coordinate[netnode] (n5) (-1.34171407598,5.95105537458) coordinate[netnode] (n6) (2.0758861129,-3.595689675) coordinate[netnode] (n7) (-6.04119876802,-1.06253756873) coordinate[netnode] (n8) (-1.37359793813,-5.95783698463) coordinate[netnode] (n9) (-2.05612586019,-3.73415858111) coordinate[netnode] (n10) (6.3509586276,-0.0416470987682) coordinate[netnode] (n11) (-4.30143454416,-4.09863362309) coordinate[netnode] (n12) (-2.25569745535,3.90707185545) coordinate[netnode] (n13) (-6.27206504806,0.823498351809) coordinate[netnode] (n14) (5.30372947143,3.47871124844) coordinate[netnode] (n15) (2.42292817113,5.84345970869) coordinate[netnode] (n16) (-3.81735339875,-1.91593227543) coordinate[netnode] (n17) (-1.73178248606,-1.64173770211) coordinate[netnode] (n18) (0.588091421565,6.10312705637) coordinate[netnode] (n19) (2.17912942633,-5.70194175276) coordinate[netnode] (n20) (2.03091755798,3.92679719021) coordinate[netnode] (n21) (3.8254204051,-2.28856060577) coordinate[netnode] (n22) (3.94086224942,4.70049587844) coordinate[netnode] (n23) (-5.57947267348,2.54232416471) coordinate[netnode] (n24) (0.00591003467172,-0.0102629739174) coordinate[netnode] (n25) (-0.124381953155,4.14846517207) coordinate[netnode] (n26) (5.28809431936,-3.57713120649) coordinate[netnode] (n27) (3.84836348643,-4.7382245695) coordinate[netnode] (n28) (-1.1949128307,2.06968815649) coordinate[netnode] (n29) (-3.53043878079,2.18202531352) coordinate[netnode] (n30) (0.916983837086,2.17886126415) coordinate[netnode] (n31) (3.56797064983,2.34784811551) coordinate[netnode] (n32) (4.26191591277,-0.0865436271976) coordinate[netnode] (n33) (-2.34545730164,0.295339841832) coordinate[netnode] (n34) (-3.20785167527,5.55624196324) coordinate[netnode] (n35) (5.84638346349,-1.78947744498) coordinate[netnode] (n36) (1.99038326562,-1.3369584776) coordinate[netnode] (n37) (-4.41618543933,0.204688715345) coordinate[netnode] (n38) (0.0690820574776,-4.4571725381) coordinate[netnode] (n39) (-4.48285212832,4.13754150742) coordinate[netnode] (n40);
\draw[link] (n38)  edge  (n14) (n38)  edge  (n17) (n38)  edge  (n24) (n38)  edge  (n34) (n39)  edge  (n4) (n39)  edge  (n5) (n39)  edge  (n7) (n39)  edge  (n10) (n39)  edge  (n18) (n30)  edge  (n13) (n30)  edge  (n24) (n30)  edge  (n29) (n30)  edge  (n34) (n30)  edge  (n38) (n30)  edge  (n40) (n31)  edge  (n2) (n31)  edge  (n21) (n31)  edge  (n25) (n31)  edge  (n26) (n31)  edge  (n29) (n31)  edge  (n32) (n32)  edge  (n3) (n32)  edge  (n15) (n32)  edge  (n33) (n33)  edge  (n3) (n33)  edge  (n11) (n33)  edge  (n37) (n34)  edge  (n17) (n34)  edge  (n25) (n35)  edge  (s14) (n35)  edge  (s15) (n7)  edge  (n37) (s44)  edge  (n27) (s44)  edge  (n36) (s45)  edge  (n27) (s45)  edge  (n36) (s46)  edge  (n11) (s46)  edge  (n36) (s47)  edge  (n11) (s47)  edge  (n36) (n12)  edge  (s28) (n12)  edge  (s29) (n12)  edge  (s30) (n12)  edge  (s31) (n12)  edge  (n10) (n12)  edge  (n17) (n13)  edge  (n6) (n13)  edge  (n26) (n13)  edge  (n29) (n13)  edge  (n35) (n13)  edge  (n40) (n10)  edge  (n4) (n10)  edge  (n17) (n10)  edge  (n18) (n11)  edge  (n36) (s7)  edge  (n16) (s24)  edge  (n14) (s4)  edge  (n15) (n18)  edge  (n4) (n18)  edge  (n17) (n18)  edge  (n25) (n18)  edge  (n34) (n19)  edge  (n26) (s19)  edge  (n24) (s18)  edge  (n35) (s13)  edge  (n6) (s13)  edge  (n19) (s12)  edge  (n6) (s12)  edge  (n19) (s11)  edge  (n16) (s11)  edge  (n19) (s10)  edge  (n16) (s10)  edge  (n19) (s17)  edge  (n35) (s16)  edge  (n35) (s38)  edge  (n5) (s35)  edge  (n5) (s35)  edge  (n9) (s34)  edge  (n5) (s34)  edge  (n9) (s37)  edge  (n5) (s37)  edge  (n20) (s36)  edge  (n5) (s36)  edge  (n20) (s22)  edge  (n14) (s22)  edge  (n24) (s9)  edge  (n16) (s9)  edge  (n19) (s8)  edge  (n16) (s23)  edge  (n14) (s3)  edge  (n3) (s3)  edge  (n15) (s2)  edge  (n3) (s2)  edge  (n15) (s1)  edge  (n3) (s1)  edge  (n11) (s0)  edge  (n3) (s0)  edge  (n11) (s6)  edge  (n15) (s5)  edge  (n15) (n40)  edge  (s16) (n40)  edge  (s17) (n40)  edge  (s18) (n40)  edge  (s19) (n29)  edge  (n25) (n29)  edge  (n26) (n29)  edge  (n34) (n28)  edge  (s39) (n28)  edge  (s40) (n28)  edge  (s41) (n28)  edge  (s42) (n28)  edge  (s43) (n28)  edge  (n7) (n23)  edge  (s4) (n23)  edge  (s5) (n23)  edge  (s6) (n23)  edge  (s7) (n23)  edge  (s8) (n23)  edge  (n21) (n22)  edge  (n7) (n22)  edge  (n27) (n22)  edge  (n33) (n22)  edge  (n36) (n22)  edge  (n37) (n21)  edge  (n16) (n21)  edge  (n26) (n21)  edge  (n32) (n20)  edge  (s38) (n20)  edge  (s39) (n20)  edge  (s40) (n20)  edge  (n7) (n27)  edge  (s41) (n27)  edge  (s42) (n27)  edge  (s43) (n26)  edge  (n6) (n25)  edge  (n37) (n8)  edge  (s23) (n8)  edge  (s24) (n8)  edge  (s25) (n8)  edge  (s26) (n8)  edge  (s27) (n8)  edge  (n38) (n9)  edge  (s32) (n9)  edge  (s33) (n9)  edge  (n39) (s20)  edge  (n14) (s20)  edge  (n24) (s21)  edge  (n14) (s21)  edge  (n24) (n0)  edge  (s25) (n0)  edge  (s26) (n0)  edge  (s27) (n0)  edge  (s28) (n0)  edge  (s29) (n0)  edge  (n17) (n1)  edge  (s30) (n1)  edge  (s31) (n1)  edge  (s32) (n1)  edge  (s33) (n1)  edge  (n9) (n1)  edge  (n10) (n2)  edge  (n21) (n2)  edge  (n25) (n2)  edge  (n32) (n2)  edge  (n33) (n2)  edge  (n37) (n4)  edge  (n7) (n4)  edge  (n25) (n4)  edge  (n37) (n6)  edge  (s14) (s15)  edge  (n6);
%Named coordinates for starting locations
\path coordinate (Spl1se0pa0) at (7.86602540378,0.5) coordinate (Spl1se0pa1) at (7.59807621135,1.5) coordinate (Spl1se0pa2) at (7.33012701892,2.5) coordinate (Spl1se0pa3) at (7.06217782649,3.5) ;
\path coordinate (Spl2se0pa0) at (6.56217782649,4.36602540378) coordinate (Spl2se0pa1) at (5.83012701892,5.09807621135) coordinate (Spl2se0pa2) at (5.09807621135,5.83012701892) coordinate (Spl2se0pa3) at (4.36602540378,6.56217782649) ;
\path coordinate (Spl3se0pa0) at (3.5,7.06217782649) coordinate (Spl3se0pa1) at (2.5,7.33012701892) coordinate (Spl3se0pa2) at (1.5,7.59807621135) coordinate (Spl3se0pa3) at (0.5,7.86602540378) ;
\path coordinate (Spl4se0pa0) at (-0.5,7.86602540378) coordinate (Spl4se0pa1) at (-1.5,7.59807621135) coordinate (Spl4se0pa2) at (-2.5,7.33012701892) coordinate (Spl4se0pa3) at (-3.5,7.06217782649) ;
\path coordinate (Spl1se1pa0) at (-4.36602540378,6.56217782649) coordinate (Spl1se1pa1) at (-5.09807621135,5.83012701892) coordinate (Spl1se1pa2) at (-5.83012701892,5.09807621135) coordinate (Spl1se1pa3) at (-6.56217782649,4.36602540378) ;
\path coordinate (Spl2se1pa0) at (-7.06217782649,3.5) coordinate (Spl2se1pa1) at (-7.33012701892,2.5) coordinate (Spl2se1pa2) at (-7.59807621135,1.5) coordinate (Spl2se1pa3) at (-7.86602540378,0.5) ;
\path coordinate (Spl3se1pa0) at (-7.86602540378,-0.5) coordinate (Spl3se1pa1) at (-7.59807621135,-1.5) coordinate (Spl3se1pa2) at (-7.33012701892,-2.5) coordinate (Spl3se1pa3) at (-7.06217782649,-3.5) ;
\path coordinate (Spl4se1pa0) at (-6.56217782649,-4.36602540378) coordinate (Spl4se1pa1) at (-5.83012701892,-5.09807621135) coordinate (Spl4se1pa2) at (-5.09807621135,-5.83012701892) coordinate (Spl4se1pa3) at (-4.36602540378,-6.56217782649) ;
\path coordinate (Spl1se2pa0) at (-3.5,-7.06217782649) coordinate (Spl1se2pa1) at (-2.5,-7.33012701892) coordinate (Spl1se2pa2) at (-1.5,-7.59807621135) coordinate (Spl1se2pa3) at (-0.5,-7.86602540378) ;
\path coordinate (Spl2se2pa0) at (0.5,-7.86602540378) coordinate (Spl2se2pa1) at (1.5,-7.59807621135) coordinate (Spl2se2pa2) at (2.5,-7.33012701892) coordinate (Spl2se2pa3) at (3.5,-7.06217782649) ;
\path coordinate (Spl3se2pa0) at (4.36602540378,-6.56217782649) coordinate (Spl3se2pa1) at (5.09807621135,-5.83012701892) coordinate (Spl3se2pa2) at (5.83012701892,-5.09807621135) coordinate (Spl3se2pa3) at (6.56217782649,-4.36602540378) ;
\path coordinate (Spl4se2pa0) at (7.06217782649,-3.5) coordinate (Spl4se2pa1) at (7.33012701892,-2.5) coordinate (Spl4se2pa2) at (7.59807621135,-1.5) coordinate (Spl4se2pa3) at (7.86602540378,-0.5) ;

  \fill[white,opacity=0.75] (-9,-9) rectangle (9,9);
  \foreach \player in {1,2,3,4} {
    \foreach \pa/\sh in {0/diamond, 1/star, 2/triang, 3/starplus} {
      \foreach \se in {0,1,2} {
        \pawn{\player}{\sh}{Spl\player se\se pa\pa}
      }
    }
  }
  \end{scope}
  \draw (-2,-2) rectangle (2,2);
\end{tikzpicture}
\caption{Pawn arrangment}
\label{fig:pawn-arrangment}
\end{figure}

\newpage

\noindent
\begin{tikzpicture}[remember picture, overlay,shift={(current page.center)}]
  \begin{scope}[yshift=9.5cm]
    %This file has been generated by POCcostFunction.py

%Places for cards
\matrix[column sep=5mm]{
\node[card place,draw=player1] {Player 1}; & \node[card place,draw=player2] {Player 2}; & \node[card place,draw=player3] {Player 3}; & \node[card place,draw=player4] {Player 4};\\
};

  \end{scope}
  \begin{scope}[yshift=-2.5cm]
    %This file has been generated by POCcostFunction.py

%Player sector colors
\path[fill=player1](8.8,0.0) -- (7.6210235533,4.4) -- (6.92820323028,4.0) -- (8.0,0.0) -- cycle;
\path[fill=player1](-4.4,7.6210235533) -- (-7.6210235533,4.4) -- (-6.92820323028,4.0) -- (-4.0,6.92820323028) -- cycle;
\path[fill=player1](-4.4,-7.6210235533) -- (-1.61653377487e-15,-8.8) -- (-1.46957615898e-15,-8.0) -- (-4.0,-6.92820323028) -- cycle;
\path[fill=player2](7.6210235533,4.4) -- (4.4,7.6210235533) -- (4.0,6.92820323028) -- (6.92820323028,4.0) -- cycle;
\path[fill=player2](-7.6210235533,4.4) -- (-8.8,1.07768918325e-15) -- (-8.0,9.79717439318e-16) -- (-6.92820323028,4.0) -- cycle;
\path[fill=player2](-1.61653377487e-15,-8.8) -- (4.4,-7.6210235533) -- (4.0,-6.92820323028) -- (-1.46957615898e-15,-8.0) -- cycle;
\path[fill=player3](4.4,7.6210235533) -- (5.38844591625e-16,8.8) -- (4.89858719659e-16,8.0) -- (4.0,6.92820323028) -- cycle;
\path[fill=player3](-8.8,1.07768918325e-15) -- (-7.6210235533,-4.4) -- (-6.92820323028,-4.0) -- (-8.0,9.79717439318e-16) -- cycle;
\path[fill=player3](4.4,-7.6210235533) -- (7.6210235533,-4.4) -- (6.92820323028,-4.0) -- (4.0,-6.92820323028) -- cycle;
\path[fill=player4](5.38844591625e-16,8.8) -- (-4.4,7.6210235533) -- (-4.0,6.92820323028) -- (4.89858719659e-16,8.0) -- cycle;
\path[fill=player4](-7.6210235533,-4.4) -- (-4.4,-7.6210235533) -- (-4.0,-6.92820323028) -- (-6.92820323028,-4.0) -- cycle;
\path[fill=player4](7.6210235533,-4.4) -- (8.8,0.0) -- (8.0,0.0) -- (6.92820323028,-4.0) -- cycle;
%Polygon
\draw[polygon] (8.0,0.0) -- (6.92820323028,4.0) -- (4.0,6.92820323028) -- (4.89858719659e-16,8.0) -- (-4.0,6.92820323028) -- (-6.92820323028,4.0) -- (-8.0,9.79717439318e-16) -- (-6.92820323028,-4.0) -- (-4.0,-6.92820323028) -- (-1.46957615898e-15,-8.0) -- (4.0,-6.92820323028) -- (6.92820323028,-4.0) -- cycle;
%Nodes and links
\path[start] (7.86602540378,0.5) coordinate[start] (s0) (7.59807621135,1.5) coordinate[start] (s1) (7.33012701892,2.5) coordinate[start] (s2) (7.06217782649,3.5) coordinate[start] (s3) (6.56217782649,4.36602540378) coordinate[start] (s4) (5.83012701892,5.09807621135) coordinate[start] (s5) (5.09807621135,5.83012701892) coordinate[start] (s6) (4.36602540378,6.56217782649) coordinate[start] (s7) (3.5,7.06217782649) coordinate[start] (s8) (2.5,7.33012701892) coordinate[start] (s9) (1.5,7.59807621135) coordinate[start] (s10) (0.5,7.86602540378) coordinate[start] (s11) (-0.5,7.86602540378) coordinate[start] (s12) (-1.5,7.59807621135) coordinate[start] (s13) (-2.5,7.33012701892) coordinate[start] (s14) (-3.5,7.06217782649) coordinate[start] (s15) (-4.36602540378,6.56217782649) coordinate[start] (s16) (-5.09807621135,5.83012701892) coordinate[start] (s17) (-5.83012701892,5.09807621135) coordinate[start] (s18) (-6.56217782649,4.36602540378) coordinate[start] (s19) (-7.06217782649,3.5) coordinate[start] (s20) (-7.33012701892,2.5) coordinate[start] (s21) (-7.59807621135,1.5) coordinate[start] (s22) (-7.86602540378,0.5) coordinate[start] (s23) (-7.86602540378,-0.5) coordinate[start] (s24) (-7.59807621135,-1.5) coordinate[start] (s25) (-7.33012701892,-2.5) coordinate[start] (s26) (-7.06217782649,-3.5) coordinate[start] (s27) (-6.56217782649,-4.36602540378) coordinate[start] (s28) (-5.83012701892,-5.09807621135) coordinate[start] (s29) (-5.09807621135,-5.83012701892) coordinate[start] (s30) (-4.36602540378,-6.56217782649) coordinate[start] (s31) (-3.5,-7.06217782649) coordinate[start] (s32) (-2.5,-7.33012701892) coordinate[start] (s33) (-1.5,-7.59807621135) coordinate[start] (s34) (-0.5,-7.86602540378) coordinate[start] (s35) (0.5,-7.86602540378) coordinate[start] (s36) (1.5,-7.59807621135) coordinate[start] (s37) (2.5,-7.33012701892) coordinate[start] (s38) (3.5,-7.06217782649) coordinate[start] (s39) (4.36602540378,-6.56217782649) coordinate[start] (s40) (5.09807621135,-5.83012701892) coordinate[start] (s41) (5.83012701892,-5.09807621135) coordinate[start] (s42) (6.56217782649,-4.36602540378) coordinate[start] (s43) (7.06217782649,-3.5) coordinate[start] (s44) (7.33012701892,-2.5) coordinate[start] (s45) (7.59807621135,-1.5) coordinate[start] (s46) (7.86602540378,-0.5) coordinate[start] (s47);
\path (-5.66458950116,-2.85371291234) coordinate[netnode] (n0) (-3.13953676463,-5.52085651601) coordinate[netnode] (n1) (2.28766112508,0.678831079224) coordinate[netnode] (n2) (5.70027506402,1.67571087853) coordinate[netnode] (n3) (0.162584255511,-2.39218233072) coordinate[netnode] (n4) (0.453712276129,-6.3681749908) coordinate[netnode] (n5) (-1.34171407598,5.95105537458) coordinate[netnode] (n6) (2.0758861129,-3.595689675) coordinate[netnode] (n7) (-6.04119876802,-1.06253756873) coordinate[netnode] (n8) (-1.37359793813,-5.95783698463) coordinate[netnode] (n9) (-2.05612586019,-3.73415858111) coordinate[netnode] (n10) (6.3509586276,-0.0416470987682) coordinate[netnode] (n11) (-4.30143454416,-4.09863362309) coordinate[netnode] (n12) (-2.25569745535,3.90707185545) coordinate[netnode] (n13) (-6.27206504806,0.823498351809) coordinate[netnode] (n14) (5.30372947143,3.47871124844) coordinate[netnode] (n15) (2.42292817113,5.84345970869) coordinate[netnode] (n16) (-3.81735339875,-1.91593227543) coordinate[netnode] (n17) (-1.73178248606,-1.64173770211) coordinate[netnode] (n18) (0.588091421565,6.10312705637) coordinate[netnode] (n19) (2.17912942633,-5.70194175276) coordinate[netnode] (n20) (2.03091755798,3.92679719021) coordinate[netnode] (n21) (3.8254204051,-2.28856060577) coordinate[netnode] (n22) (3.94086224942,4.70049587844) coordinate[netnode] (n23) (-5.57947267348,2.54232416471) coordinate[netnode] (n24) (0.00591003467172,-0.0102629739174) coordinate[netnode] (n25) (-0.124381953155,4.14846517207) coordinate[netnode] (n26) (5.28809431936,-3.57713120649) coordinate[netnode] (n27) (3.84836348643,-4.7382245695) coordinate[netnode] (n28) (-1.1949128307,2.06968815649) coordinate[netnode] (n29) (-3.53043878079,2.18202531352) coordinate[netnode] (n30) (0.916983837086,2.17886126415) coordinate[netnode] (n31) (3.56797064983,2.34784811551) coordinate[netnode] (n32) (4.26191591277,-0.0865436271976) coordinate[netnode] (n33) (-2.34545730164,0.295339841832) coordinate[netnode] (n34) (-3.20785167527,5.55624196324) coordinate[netnode] (n35) (5.84638346349,-1.78947744498) coordinate[netnode] (n36) (1.99038326562,-1.3369584776) coordinate[netnode] (n37) (-4.41618543933,0.204688715345) coordinate[netnode] (n38) (0.0690820574776,-4.4571725381) coordinate[netnode] (n39) (-4.48285212832,4.13754150742) coordinate[netnode] (n40);
\draw[link] (n38)  edge  (n14) (n38)  edge  (n17) (n38)  edge  (n24) (n38)  edge  (n34) (n39)  edge  (n4) (n39)  edge  (n5) (n39)  edge  (n7) (n39)  edge  (n10) (n39)  edge  (n18) (n30)  edge  (n13) (n30)  edge  (n24) (n30)  edge  (n29) (n30)  edge  (n34) (n30)  edge  (n38) (n30)  edge  (n40) (n31)  edge  (n2) (n31)  edge  (n21) (n31)  edge  (n25) (n31)  edge  (n26) (n31)  edge  (n29) (n31)  edge  (n32) (n32)  edge  (n3) (n32)  edge  (n15) (n32)  edge  (n33) (n33)  edge  (n3) (n33)  edge  (n11) (n33)  edge  (n37) (n34)  edge  (n17) (n34)  edge  (n25) (n35)  edge  (s14) (n35)  edge  (s15) (n7)  edge  (n37) (s44)  edge  (n27) (s44)  edge  (n36) (s45)  edge  (n27) (s45)  edge  (n36) (s46)  edge  (n11) (s46)  edge  (n36) (s47)  edge  (n11) (s47)  edge  (n36) (n12)  edge  (s28) (n12)  edge  (s29) (n12)  edge  (s30) (n12)  edge  (s31) (n12)  edge  (n10) (n12)  edge  (n17) (n13)  edge  (n6) (n13)  edge  (n26) (n13)  edge  (n29) (n13)  edge  (n35) (n13)  edge  (n40) (n10)  edge  (n4) (n10)  edge  (n17) (n10)  edge  (n18) (n11)  edge  (n36) (s7)  edge  (n16) (s24)  edge  (n14) (s4)  edge  (n15) (n18)  edge  (n4) (n18)  edge  (n17) (n18)  edge  (n25) (n18)  edge  (n34) (n19)  edge  (n26) (s19)  edge  (n24) (s18)  edge  (n35) (s13)  edge  (n6) (s13)  edge  (n19) (s12)  edge  (n6) (s12)  edge  (n19) (s11)  edge  (n16) (s11)  edge  (n19) (s10)  edge  (n16) (s10)  edge  (n19) (s17)  edge  (n35) (s16)  edge  (n35) (s38)  edge  (n5) (s35)  edge  (n5) (s35)  edge  (n9) (s34)  edge  (n5) (s34)  edge  (n9) (s37)  edge  (n5) (s37)  edge  (n20) (s36)  edge  (n5) (s36)  edge  (n20) (s22)  edge  (n14) (s22)  edge  (n24) (s9)  edge  (n16) (s9)  edge  (n19) (s8)  edge  (n16) (s23)  edge  (n14) (s3)  edge  (n3) (s3)  edge  (n15) (s2)  edge  (n3) (s2)  edge  (n15) (s1)  edge  (n3) (s1)  edge  (n11) (s0)  edge  (n3) (s0)  edge  (n11) (s6)  edge  (n15) (s5)  edge  (n15) (n40)  edge  (s16) (n40)  edge  (s17) (n40)  edge  (s18) (n40)  edge  (s19) (n29)  edge  (n25) (n29)  edge  (n26) (n29)  edge  (n34) (n28)  edge  (s39) (n28)  edge  (s40) (n28)  edge  (s41) (n28)  edge  (s42) (n28)  edge  (s43) (n28)  edge  (n7) (n23)  edge  (s4) (n23)  edge  (s5) (n23)  edge  (s6) (n23)  edge  (s7) (n23)  edge  (s8) (n23)  edge  (n21) (n22)  edge  (n7) (n22)  edge  (n27) (n22)  edge  (n33) (n22)  edge  (n36) (n22)  edge  (n37) (n21)  edge  (n16) (n21)  edge  (n26) (n21)  edge  (n32) (n20)  edge  (s38) (n20)  edge  (s39) (n20)  edge  (s40) (n20)  edge  (n7) (n27)  edge  (s41) (n27)  edge  (s42) (n27)  edge  (s43) (n26)  edge  (n6) (n25)  edge  (n37) (n8)  edge  (s23) (n8)  edge  (s24) (n8)  edge  (s25) (n8)  edge  (s26) (n8)  edge  (s27) (n8)  edge  (n38) (n9)  edge  (s32) (n9)  edge  (s33) (n9)  edge  (n39) (s20)  edge  (n14) (s20)  edge  (n24) (s21)  edge  (n14) (s21)  edge  (n24) (n0)  edge  (s25) (n0)  edge  (s26) (n0)  edge  (s27) (n0)  edge  (s28) (n0)  edge  (s29) (n0)  edge  (n17) (n1)  edge  (s30) (n1)  edge  (s31) (n1)  edge  (s32) (n1)  edge  (s33) (n1)  edge  (n9) (n1)  edge  (n10) (n2)  edge  (n21) (n2)  edge  (n25) (n2)  edge  (n32) (n2)  edge  (n33) (n2)  edge  (n37) (n4)  edge  (n7) (n4)  edge  (n25) (n4)  edge  (n37) (n6)  edge  (s14) (s15)  edge  (n6);
%Named coordinates for starting locations
\path coordinate (Spl1se0pa0) at (7.86602540378,0.5) coordinate (Spl1se0pa1) at (7.59807621135,1.5) coordinate (Spl1se0pa2) at (7.33012701892,2.5) coordinate (Spl1se0pa3) at (7.06217782649,3.5) ;
\path coordinate (Spl2se0pa0) at (6.56217782649,4.36602540378) coordinate (Spl2se0pa1) at (5.83012701892,5.09807621135) coordinate (Spl2se0pa2) at (5.09807621135,5.83012701892) coordinate (Spl2se0pa3) at (4.36602540378,6.56217782649) ;
\path coordinate (Spl3se0pa0) at (3.5,7.06217782649) coordinate (Spl3se0pa1) at (2.5,7.33012701892) coordinate (Spl3se0pa2) at (1.5,7.59807621135) coordinate (Spl3se0pa3) at (0.5,7.86602540378) ;
\path coordinate (Spl4se0pa0) at (-0.5,7.86602540378) coordinate (Spl4se0pa1) at (-1.5,7.59807621135) coordinate (Spl4se0pa2) at (-2.5,7.33012701892) coordinate (Spl4se0pa3) at (-3.5,7.06217782649) ;
\path coordinate (Spl1se1pa0) at (-4.36602540378,6.56217782649) coordinate (Spl1se1pa1) at (-5.09807621135,5.83012701892) coordinate (Spl1se1pa2) at (-5.83012701892,5.09807621135) coordinate (Spl1se1pa3) at (-6.56217782649,4.36602540378) ;
\path coordinate (Spl2se1pa0) at (-7.06217782649,3.5) coordinate (Spl2se1pa1) at (-7.33012701892,2.5) coordinate (Spl2se1pa2) at (-7.59807621135,1.5) coordinate (Spl2se1pa3) at (-7.86602540378,0.5) ;
\path coordinate (Spl3se1pa0) at (-7.86602540378,-0.5) coordinate (Spl3se1pa1) at (-7.59807621135,-1.5) coordinate (Spl3se1pa2) at (-7.33012701892,-2.5) coordinate (Spl3se1pa3) at (-7.06217782649,-3.5) ;
\path coordinate (Spl4se1pa0) at (-6.56217782649,-4.36602540378) coordinate (Spl4se1pa1) at (-5.83012701892,-5.09807621135) coordinate (Spl4se1pa2) at (-5.09807621135,-5.83012701892) coordinate (Spl4se1pa3) at (-4.36602540378,-6.56217782649) ;
\path coordinate (Spl1se2pa0) at (-3.5,-7.06217782649) coordinate (Spl1se2pa1) at (-2.5,-7.33012701892) coordinate (Spl1se2pa2) at (-1.5,-7.59807621135) coordinate (Spl1se2pa3) at (-0.5,-7.86602540378) ;
\path coordinate (Spl2se2pa0) at (0.5,-7.86602540378) coordinate (Spl2se2pa1) at (1.5,-7.59807621135) coordinate (Spl2se2pa2) at (2.5,-7.33012701892) coordinate (Spl2se2pa3) at (3.5,-7.06217782649) ;
\path coordinate (Spl3se2pa0) at (4.36602540378,-6.56217782649) coordinate (Spl3se2pa1) at (5.09807621135,-5.83012701892) coordinate (Spl3se2pa2) at (5.83012701892,-5.09807621135) coordinate (Spl3se2pa3) at (6.56217782649,-4.36602540378) ;
\path coordinate (Spl4se2pa0) at (7.06217782649,-3.5) coordinate (Spl4se2pa1) at (7.33012701892,-2.5) coordinate (Spl4se2pa2) at (7.59807621135,-1.5) coordinate (Spl4se2pa3) at (7.86602540378,-0.5) ;

  \end{scope}
\end{tikzpicture}

\newpage
\section*{Pawns}
\begin{tikzpicture}
  \newcommand{\colstride}{4.5}
  \newcommand{\rowstride}{1.5}
  \foreach \p in {1,2,3,4} {
    \foreach \xx/\sh in {0/diamond, 1/star, 2/triang, 3/starplus} {
      \foreach \x in {0,1,2,3} {
        \pawn{\p}{\sh}{\colstride*\xx+\x,-\rowstride*\p}
      }
    }
  }
\end{tikzpicture}
\section*{Letter tokens}
\begin{tikzpicture}
  \newcommand{\colstridetoken}{2cm}
  \newcommand{\rowstridetoken}{2cm}
  \newcommand{\tokensize}{\cardwidth/2.2}
  \foreach \p in {1,2,3,4} {
    \foreach \xx/\sh in {0/diamond,1/star,2/triang,3/starplus} {
      \node[circle,fill=player\p,minimum size=\tokensize] at (\colstridetoken*\xx,-\rowstridetoken*\p) {};
      \node[inner sep=0pt,\sh,fill=white,minimum size=0.8*\tokensize] at (\colstridetoken*\xx,-\rowstridetoken*\p) {};
    }
  }
\end{tikzpicture}

\section*{Word cards}
See next pages.

\tikzset{ 
  table/.style={
    matrix of nodes,
    row sep=-\pgflinewidth,
    column sep=-\pgflinewidth,
    nodes={
      rectangle,
      draw=black,
      align=center
    },
    minimum height=\cardheight,
    minimum width=\cardwidth,
    text depth=0.5ex,
    text height=2ex,
    nodes in empty cells,
    font=\sffamily
  }
}

\newpage\noindent
\begin{tikzpicture}
\matrix[table]{
TEST & WITH & HILL & BIRD & CORN\\
BEAR & SURE & MOST & GRAY & FINE\\
FIND & KILL & HOLE & FORM & ABLE\\
PART & LAND & ROSE & TONE & FIRE\\
THIN & WENT & SING & FALL & SHOP\\
};
\end{tikzpicture}
\begin{tikzpicture}
\matrix[table]{
DUCK & JUMP & ONCE & ROOM & LADY\\
LONG & PLAN & WEST & SOFT & KEEP\\
HERE & TERM & LEAD & GONE & PULL\\
PATH & EVEN & INCH & STAY & BANK\\
KEPT & MISS & BASE & CROP & THEN\\
};
\end{tikzpicture}
\begin{tikzpicture}
\matrix[table]{
CITY & EASE & KNOW & FEEL & WEEK\\
WOOD & FELT & HUGE & FLAT & LAST\\
OVER & CAMP & LATE & PORT & SEND\\
NEAR & MUCH & FOOD & RAIN & IDEA\\
GIVE & SHOW & FEAR & BURN & SEED\\
};
\end{tikzpicture}


\section*{Acknowledgements}
The development of \gamename{} was made possible by the support of grant IIS-1717656 by the National Science Foundation.

\includegraphics[width=3cm]{figures/NSF-4-Color-bitmap-Logo}

\end{document}
