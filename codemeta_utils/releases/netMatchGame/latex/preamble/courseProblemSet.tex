% Make font for captions smaller (saves space and visually separates them from the main text)
\usepackage[font={footnotesize,sf}]{caption}

% Print menu and key command descriptions
% \usepackage{menukeys}

% make underscore work without escaping in text mode
\usepackage{underscore}

% Common style that will be used for titles
\newcommand{\titlestyle}{\color{DodgerBlue3}\sffamily}

% Perfect 1-inch margins
\usepackage[margin=1in]{geometry}
\usepackage[medium,compact]{titlesec}
\usepackage{longtable}
\titleformat*{\section}{\titlestyle \Large}
\titleformat*{\subsection}{\titlestyle \large \itshape}

% Lists (requires enumitem package):
% Font for description
\setlist[description]{font=\color{DodgerBlue3}}
% Compact lists
\setlist[1]{\labelindent=\parindent}
\setlist[1]{topsep=3pt, itemsep=0pt}

% Format main title
\let\prevtitle\title
\renewcommand{\title}[1]{\prevtitle{\titlestyle #1}}
\let\prevauthor\author
\renewcommand{\author}[1]{\prevauthor{\titlestyle #1}}
\let\prevdate\date
\renewcommand{\date}[1]{\prevdate{\titlestyle #1}}

% Displayed quotations (used to define an environment for function specification)
\usepackage{csquotes}

% Include code
% Include code and org files
\usepackage{listings}
% Smaller fixed-width font, automatic breaking of long lines, ignore indentation in list environments
\lstset{basicstyle=\footnotesize\ttfamily,breaklines=true,resetmargins=true}


% formatting for the description of a function's input/output arguments
\mdfdefinestyle{functionspecification}{%
  linecolor=DodgerBlue3,
  rightline=false, topline=false, bottomline=false,
  outerlinewidth=1.35,
  roundcorner=10pt,
  innerleftmargin=1.2em,
  innertopmargin=0.3em,
  innerbottommargin=0.3em,
  leftmargin=1pt,
  skipabove=0.5em,
  skipbelow=0.5em,
  backgroundcolor=white}

\newcommand{\dstar}{$\bigstar$}

\newif\ifmatlab
\matlabtrue
% Enable conditional part based on whether the language used is Matlab or not
% \ifmatlab
% % use Matlab
% \else
% % use Python or something else
% \fi

\newcommand{\hwlanguage}{\ifmatlab{}Matlab\else{}Python\fi{}}
\newcommand{\ifmtlb}[2]{\ifmatlab{}#1\else{}#2\fi{}}
\newcommand{\vartrue}{\var{\ifmatlab{}true\else{}True\fi{}}}
\newcommand{\varfalse}{\var{\ifmatlab{}false\else{}False\fi{}}}
\newcommand{\varlang}[2]{\var{\ifmtlb{#1}{#2}}}

\newcommand{\langfield}[1]{\ifmtlb{.#1}{['#1']}}
\newcommand{\functionlanguage}[3]{\function{\ifmatlab{}#1\else{}#2\fi{}}{#3}}
\newcommand{\functionheaderlanguage}[4][]{%
  \ifmatlab{}%
  \functionheader[#1]{#2}{#4}%
  \else{}%
  \fname[Method]{#3}%
  \fi{}%
}
\newenvironment{functionspeclang}[1]
{\functionspecification\ifmatlab{}\else{}\ffilename{#1}\fi{}}
{\endfunctionspecification}

\NewEnviron{functionclassspeclang}[3][]{%
  \ifmatlab{}%
  \begin{functionspecification}
    \BODY%
  \end{functionspecification}
  \else{}%
  \begin{classspecification}%
    \cname{#2}%
    \ffilename{#3}%
    \ifnotempty{#1}{
      \begin{cdescription}
        #1
      \end{cdescription}
    }
    \begin{functionspecification}
      \BODY%
    \end{functionspecification}
  \end{classspecification}%
  \fi{}%
}
\newcommand{\hwtitle}[2]{
\ifmatlab
\title{Homework #2 (Matlab version)}
\else
\title{Homework #2 (Python version)}
\fi
\author{#1 - Prof. Tron}
\date{\today}
\edef\hwnumber{#2}
}


%Facilities for markers file
% open file the beginning of the document
\AtBeginDocument{%
\newwrite\markerstream
\immediate\openout\markerstream=\jobname.mlg
}
% close the file at the end of the document
\AtEndDocument{%
\immediate\closeout\markerstream
}
%definition similar to \typeout, but for markers
\makeatletter
\newcommand{\markerout}[1]{\begingroup \set@display@protect \immediate \write \markerstream {#1}\endgroup}
\makeatother

\usetikzlibrary{intersections}

\titleformat{\section}{\titlestyle\large\bfseries}{Problem \thesection:}{0.5ex}{}
%\titleformat{\subsection}{\normalfont\bfseries}{Problem \thesubsection:}{0.5ex}{}
\titleformat{\paragraph}[runin]{\titlestyle\normalsize\bfseries}{}{2pt}{}

%Macro for paragraph header for preparation steps to questions
%\newcommand{\preparation}{\smallskip\par\noindent\textbf{\titlestyle Preparation.~}}

\newcommand{\preparation}[1][]{%
\smallskip\par\noindent%
\textbf{\titlestyle Preparation\displayifnotempty{~(}{#1}{)}.~}}

%New custom counter for question
\newcounter{question}[section]

%Macro for writing points for the current question to the log file.
\newcommand{\pointmarker}{\typeout{Question points marker [\thesection.\thequestion] [\questionpoints]}}

\makeatletter
%Makes questions that:
%- are automatically numbered using the format <sectionNumber>.<questionNumber>
%- can be referred to using the usual \label-\ref mechanism
%- if the optional argument is not empty, this is displayed in parenthesis (typically, this will be a \points{n} or a \optional command). If the optional argument is empty, a single point is silently added to the point counter.
\newcommand{\question}[1][]{%
\smallskip\par\noindent%
\refstepcounter{question}\zerovar{\questionpoints}%
\sbox\boxifnotempty{\textbf{\titlestyle #1}}% for some reason, if the box is saved inside the \textbf{\titlestyle ...} (to avoid repeating the style), the box counts as empty for \ifempty, so I need to repeat the style here.
\textbf{\titlestyle Question \thesection.\thequestion\displayifnotempty{~(}{\usebox{\boxifnotempty}}{)}.~}%
\ifempty{\usebox{\boxifnotempty}}{\increasepoints{1}}%
\def\@currentlabel{{\thesection.\thequestion}}\pointmarker%
\markerout{Marker question [nolabel] [\thesection.\thequestion]}}%

%Commands that defines a new `labeled' question, which uses a separate counter but the same total point count. A new label with the same name must be defined separately
\newcommand{\newquestionlabel}[1]{%
\newcounter{question#1}[section]%
\expandafter\newcommand\csname question#1\endcsname[1][]{%
\smallskip\par\noindent%
\refstepcounter{question#1}\zerovar{\questionpoints}%
\sbox\boxifnotempty{\textbf{\titlestyle ##1}}% for some reason, if the box is saved inside the \textbf{\titlestyle ...} (to avoid repeating the style), the box counts as empty for \ifempty, so I need to repeat the style here.
\textbf{\titlestyle Question \csname #1\endcsname~\thesection.\csname thequestion#1\endcsname\displayifnotempty{~(}{\usebox{\boxifnotempty}}{)}.~}%
\ifempty{\usebox{\boxifnotempty}}{\increasepoints{1}}%
\def\@currentlabel{{#1~\thesection.\csname thequestion#1\endcsname}}\pointmarker%
\markerout{Marker question [#1] [\thesection.\csname thequestion#1\endcsname]}}%
}
\makeatother

\newquestionlabel{code}
\newquestionlabel{report}
\newquestionlabel{optional}
\newquestionlabel{provided}
\newquestionlabel{image}
\newquestionlabel{video}
\newquestionlabel{scan}
\newquestionlabel{feedback}

\newcommand{\hintquestion}[1]{\paragraph{Hint
for question \ref{#1}:~}}
\newcommand{\hint}{\paragraph{Hint:~}}

\newcommand{\answer}[1][]{\smallskip\leavevmode\\\noindent\textbf{Answer \thesection.\thequestion#1:~}}

\newcommand{\updated}[1]{{\color{red} #1}}

\newcolorlabel{labeloptional}{DodgerBlue3}
\newcommand{\optional}{\labeloptional{optional}}
\newcolorlabel{labelprovided}{Orange1}
\newcommand{\provided}{\labelprovided{provided}}
\newcolorlabel{labelvideo}{Coral2}
\newcommand{\video}{\labelvideo{video}}
\newcolorlabel{labelimage}{DarkOrchid2}
\newcommand{\image}{\labelimage{image}}
\newcolorlabel{labelcode}{SlateBlue2}
\newcommand{\code}{\labelcode{code}}
\newcolorlabel{labelscan}{Chocolate3}
\newcommand{\scan}{\labelcode{scan}}
\newcolorlabel{labelnew}{OliveDrab3}
\newcommand{\lbnew}{\labelnew{new}}
\newcolorlabel{labelreport}{OliveDrab3}
\newcommand{\report}{\labelreport{report}}
\newcolorlabel{labelgoal}{Black}
\newcommand{\goal}{\labelgoal{Goal}}
\newcolorlabel{labelfeedback}{SeaGreen4}
\newcommand{\feedback}{\labelfeedback{feedback}}

%Camera icon
\usepackage{fontawesome}
\newcommand{\screenshot}{\textcolor{DarkOrchid2}{\faCamera~Take a screenshot.}}

\tikzset{point/.style={inner sep=0pt,fill=black,circle,minimum size=4pt}}

%macros for keeping track of points (total, and for current question)
\zerovar{\totalpoints}
\zerovar{\bonuspoints}
\zerovar{\questionpoints}
\newcommand{\increasepoints}[1]{\addtovar{\totalpoints}{#1}\addtovar{\questionpoints}{#1}}
\newcommand{\points}[1]{#1 point\ifthenelse{\equal{#1}{1}}{}{s}\increasepoints{#1}}
\makeatletter
\newcommand{\totalpointswithbonus}[1]{\pgfmathparse{\storeddef{totalpoints}+#1}\pgfmathresult\typeout{Total points marker [\pgfmathresult]}}
\makeatother

\newstoreddef{totalpoints} %remember total number of points in aux file

% answer boxes
\newcommand{\vspaceboxdelta}{3mm}
\newcommand{\vspacebox}[1]{\par\bigskip{\tikz{\useasboundingbox (0,-\vspaceboxdelta) rectangle (\textwidth,#1\baselineskip); \draw[gray!20!white,line width=3pt] (-\vspaceboxdelta,-\vspaceboxdelta) rectangle (\textwidth+2*\vspaceboxdelta,#1\baselineskip+\vspaceboxdelta);}}\par}
\newcommand{\vspaceoneline}{\vspacebox{2}}
\newcommand{\vspacetwolines}{\vspacebox{3}}
\newcommand{\vspacethreelines}{\vspacebox{4}}
\newcommand{\vspacefourlines}{\vspacebox{5}}
\newcommand{\vspacefivelines}{\vspacebox{6}}
\newcommand{\vspacesixlines}{\vspacebox{7}}

% text fields
\newcommand{\textrowfield}[1]{\par\noindent#1: \enspace\hrulefill}

% formatting for the description of a function's input/output arguments
\usepackage[framemethod=tikz]{mdframed}
\mdfdefinestyle{functionspecification}{%
linecolor=DodgerBlue3,
rightline=false, topline=false, bottomline=false,
outerlinewidth=1.35,
roundcorner=10pt,
innerleftmargin=1.2em,
innertopmargin=0.3em,
innerbottommargin=0.3em,
leftmargin=1pt,
skipabove=0.5em,
skipbelow=0.5em,
backgroundcolor=white}
\mdfdefinestyle{classspecification}{%
linecolor=DarkOliveGreen3,
rightline=false, topline=false, bottomline=false,
outerlinewidth=1.35,
roundcorner=10pt,
innerleftmargin=1.2em,
innertopmargin=0.3em,
innerbottommargin=0.3em,
leftmargin=1pt,
skipabove=0.5em,
skipbelow=0.5em,
backgroundcolor=white}
%Note: I was not able to avoid duplication of options.

\newenvironment{functionspecification}{\markerout{Marker function [start]}\begin{mdframed}[style=functionspecification]}
{\markerout{Marker function [end]}\end{mdframed}}%
\newenvironment{classspecification}{\markerout{Marker class [start]}\begin{mdframed}[style=classspecification]}
{\markerout{Marker class [end]}\end{mdframed}}%
%\NewEnviron{fdescription}{\par\noindent{\titlestyle Description}: \BODY\markerout{Marker text [description] [start]^^J\BODY^^JMarker text [description] [end]}\par}
\NewEnviron{fdescription}{\par\noindent{\titlestyle Description}: \BODY\markerout{Marker text [description] [start]^^J\BODY^^JMarker text [description] [end]}\par}
\newenvironment{cdescription}{\fdescription}{\endfdescription}
\NewEnviron{frequirements}{\par\noindent{\titlestyle Requirements}: \BODY\markerout{Marker text [requirements] [start]^^J\BODY^^JMarker text [requirements] [end]}\par}
\newcommand{\cname}[1]{\noindent{\titlestyle Class name}: \var{#1}\par}
\newenvironment{finputs}{\markerout{Marker argument_group [start] [inputs]}\noindent{\titlestyle Input arguments}\begin{itemize}}{\end{itemize}\markerout{Marker argument_group [end]}}
\newenvironment{foptional}{\markerout{Marker argument_group [start] [optional]}\noindent{\titlestyle Optional arguments}\begin{itemize}}{\end{itemize}\markerout{Marker argument_group [end]}}
\newenvironment{foutputs}[1][Output]{\markerout{Marker argument_group [start] [outputs]}\noindent{\titlestyle #1 arguments}\begin{itemize}}{\end{itemize}\markerout{Marker argument_group [end]}}
\newcommand{\freturns}[2][]{\noindent{\titlestyle Returns}: \var{#2}\displayifnotempty{ (type }{\var{#1}}{)}}
\newcommand{\argument}[4][]{\markerout{Marker argument [start]}\markerout{Marker argument [name] [#2]}\var{#2}
\ifnotempty{#1#3}{(\ifnotempty{#3}{\markerout{Marker argument [dimensions] [#3] [#4]}dim. \vardim{#3}{#4}\ifnotempty{#1}{, }}%
\ifnotempty{#1}{\markerout{Marker argument [type] [#1]}type \var{#1}\markerout{Marker argument [end]}}
)}\markerout{Marker argument [end]}}%removed the \var{}
\newcommand{\argumentlist}[3][]{\markerout{Marker argument [start]}\markerout{Marker argument [name] [#2]}\var{#2} %
\ifnotempty{#1#3}{(\ifnotempty{#3}{\markerout{Marker argument [dimensions] [#3] [1]} length #3\ifnotempty{#1}{, }}%
\displayifnotempty{\markerout{Marker argument [type] [#1]}type }{\var{#1}}{})}\markerout{Marker argument [end]}}
\newcommand{\argumentscalar}[2][]{\argument[#1]{#2}{}{}}
\newcommand{\fheader}[3][]{\noindent{\titlestyle Header}: \function[#1]{#2}{#3}\par}
\newcommand{\fname}[2][Function]{\noindent{\titlestyle #1 name}: \var{#2}\par}
\newcommand{\ffilename}[1]{\noindent{\titlestyle File name}: \file{#1}\par}
\newcommand{\functionheader}[3][]{\markerout{Marker function_header [#1] [#2] [#3]}\function[#1]{#2}{#3}\par}

% Commands to indicate functions and files that are provided but not fully specified
\newcommand{\mfunctionprovided}[3][]{\markerout{Marker file_provided [#2.m]}\function[#1]{#2}{#3}}
\newcommand{\fileprovided}[2][]{\markerout{Marker file_provided [#1#2]}\file{#2}}
\newcommand{\fileincluded}[2][]{\markerout{Marker file_provided [#1#2]}}
%note: fileprovided and fileincluded differ on whether the file name is inserted in the TeX output (in both cases, it is inserted in the marker log)

%ROS node specification
\newenvironment{nodespecification}{\begin{mdframed}[style=functionspecification]%
\newcommand{\filename}[1]{\par{\titlestyle File name:} \file{##1}}
\renewenvironment{description}{\par\noindent{\titlestyle Description:}}{\par}%
\newenvironment{subscribers}{\par\noindent{\titlestyle Subscribes to topics:}}{}%
\newenvironment{publishers}{\par\noindent{\titlestyle Publishes to topics:}}{}%
%
}{\end{mdframed}}

\newcommand{\rostopic}[2]{\var{#1} (type \var{#2})}

% Macros for questions where student need to draw on axes
\newcommand{\drawingaxes}[1][]{\draw[gray] (-5,-5) grid (5,5);
\draw[very thick, -latex] (-5.7,0) -- (5.7,0) coordinate (a1);
\draw[very thick, -latex] (0,-5.7) -- (0,5.7) coordinate (a2); #1}

\newcommand{\shifteddrawingaxes}[1][]{
\begin{scope}[xshift=18cm]
\drawingaxes[#1]
\end{scope}
}

\newcommand{\drawexercise}[2][]{\begin{center}
\begin{tikzpicture}[scale=0.6]
\node[text width=7cm,anchor=west]{#2};
\shifteddrawingaxes[{\node at ($(a1)-(0,1.2em)$) {$x_1$}; \node at ($(a2)+(1.2em,0)$) {$x_2$}; #1}]
\end{tikzpicture}
\end{center}
}

% Insert a random quote from file
% ref: https://tex.stackexchange.com/questions/187000/can-latex-select-a-random-line-from-another-file
\usepackage{datatool}
\newcommand{\quotecomma}{,{}}
\newcommand{\randomquote}[1][preamble/quotes.txt]{%
  \DTLsetseparator{|}%
  \DTLloadrawdb[keys={Number,Quote,Author}]{quotes}{#1}% Load quotes
  \edef\RandomQuote{\number\numexpr1+\pdfuniformdeviate\DTLrowcount{quotes}}% Identify random quote
  \dtlgetrow{quotes}{\RandomQuote}% Retrieve random quote
  \dtlgetentryfromcurrentrow{\Quote}{1}\dtlgetentryfromcurrentrow{\Author}{2}%
  ``\Quote''\\---~\emph{\Author}%
}
\newcommand{\randomquotebox}{
  \begin{mdframed}
    \randomquote{}
  \end{mdframed}
}

\newcommand{\copyrightnote}{\copyright\the\year Boston University and Roberto Tron. All rights reserved. No distribution without written permission.}
